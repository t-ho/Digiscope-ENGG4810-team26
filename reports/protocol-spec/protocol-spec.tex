\documentclass[]{article}

\usepackage{bytefield}
\usepackage{float}
\usepackage{hyperref}
\usepackage{textcomp}

\usepackage[a4paper, margin=1.5cm]{geometry}

%opening
\title{ENGG4810 Oscilloscope Communication Protocol Specification}
\author{
	Ryan Fitzimon\\
	43214907\\
	\texttt{ryan.fitzsimon@uq.net.au}
	\and
	Minh Toan Ho\\
	43129560\\
	\texttt{minh.ho2@uq.net.au}
}


\begin{document}

\maketitle

\section{Packet Types}

Values are signed 32 bit integers unless otherwise stated.

\begin{table}[H]
	\renewcommand{\arraystretch}{1.2}
	\centering
	\begin{tabular}{|c|c|c|}
		\hline
		\textbf{Identifier} & \textbf{Type} & \textbf{Value} \\ \hline
		0x02 & Horizontal Range & Range in \textmu s \\ \hline
		0xA1 & Channel A Vertical Range & Range in \textmu V \\ \hline
		0xB1 & Channel B Vertical Range & Range in \textmu V \\ \hline
		0xA3 & Channel A Trigger Mode & 0:Auto, 1:Single, 2:Normal \\ \hline
		0xB3 & Channel B Trigger Mode & 0:Auto, 1:Single, 2:Normal \\ \hline
		0xA4 & Channel A Trigger Type & 0:Level, 1:Rising, 2:Falling \\ \hline
		0xB4 & Channel B Trigger Type & 0:Level, 1:Rising, 2:Falling \\ \hline
		0xA5 & Channel A Trigger Threshold & Threshold in \textmu V \\ \hline
		0xB5 & Channel B Trigger Threshold & Threshold in \textmu V \\ \hline
		0xA6 & Channel A DC Offset & Offset in \textmu V \\ \hline
		0xB6 & Channel B DC Offset & Offset in \textmu V \\ \hline
		0xAA & Channel A Trigger Arm & 0:Off, Other:On \\ \hline 
		0xBA & Channel B Trigger Arm & 0:Off, Other:On \\ \hline 
		0xAF & Channel A Trigger Force & Ignored \\ \hline 
		0xBF & Channel B Trigger Force & Ignored \\ \hline 
		0xAC & Channel A Coupling & 0:DC, Other:AC \\ \hline
		0xBC & Channel B Coupling & 0:DC, Other:AC \\ \hline
		0xA7 & Channel A Mode & 0:8bit, Other:12bit \\ \hline
		0xB7 & Channel B Mode & 0:8bit, Other:12bit \\ \hline
		0xF0 & Function Generator Output & 0:Off, Other:On \\ \hline
		0xF1 & Function Generator Wave Type & 0:Sine, 1:Square, 2:Triangle, 3:Ramp, 4:Noise \\ \hline
		0xF2 & Function Generator Voltage & Voltage in \textmu V \\ \hline
		0xF3 & Function Generator Offset & Voltage in \textmu V \\ \hline
		0xF4 & Function Generator Frequency & Frequency in Hz \\ \hline
		0xEE & Keep alive & Ignored \\ \hline
		0xD0 & Display Power & 0:Off, Other:On \\ \hline
		0xD1 & Display Backlight Brightness & Brightness level (0 - 5) \\ \hline
		0xAD & 8 bit sample packet, channel A & See \autoref{8bit} \\ \hline
		0xBD & 8 bit sample packet, channel B & See \autoref{8bit} \\ \hline
		0xAE & 12 bit sample packet, channel A & See \autoref{12bit} \\ \hline
		0xBE & 12 bit sample packet, channel B & See \autoref{12bit} \\ \hline
	\end{tabular}
	\caption{Packet type identifiers}
\end{table}

\section{General Structure}
\begin{figure}[H]
	\centering
	\begin{bytefield}[bitwidth=2em]{16}
		\bitheader{0-15} \\
		\bitbox{8}{Packet Type} & \bitbox[ltr]{8}{} \\
		\wordbox[lr]{1}{\dots} \\
		\wordbox[lr]{1}{Data} \\
		\wordbox[blr]{1}{\dots}
	\end{bytefield}
	\caption{Generic packet structure}
\end{figure}

\section{Command Packet}

0x00 for command, 0xFF for confirmation.

\begin{figure}[H]
	\centering
	\begin{bytefield}[bitwidth=2em]{16}
		\bitheader{0-15} \\
		\bitbox{8}{Packet Type} & \bitbox{8}{Command / Confirmation Indicator} \\
		\wordbox{2}{Argument}
	\end{bytefield}
	\caption{Command packet structure}
\end{figure}

\section{Data Packet}

Sequence numbers begin at 0.

\subsection{8 bit mode}
\label{8bit}
\begin{figure}[H]
	\centering
	\begin{bytefield}[bitwidth=2em]{16}
		\bitheader{0-15} \\
		\bitbox{8}{Packet Type} & \bitbox{8}{Sequence Number} \\
		\bitbox{16}{Number of Samples} \\
		\bitbox{16}{Sample Period in \textmu s} \\
		\begin{rightwordgroup}{Payload}
			\bitbox{8}{Sample 0} & \bitbox{8}{Sample 1} \\
			\bitbox{8}{Sample 2} & \bitbox{8}{Sample 3}\\
			\wordbox[tlr]{1}{\dots} \\
			\wordbox[lr]{1}{Samples} \\
			\wordbox[lr]{1}{\dots} \\
			\bitbox[blr]{8}{} & \bitbox{8}{Sample n}
		\end{rightwordgroup}
	\end{bytefield}
	\caption{8 bit mode data packet structure}
\end{figure}

\subsection{12 bit mode}
\label{12bit}
\begin{figure}[H]
	\centering
	\begin{bytefield}[bitwidth=2em]{16}
		\bitheader{0-15} \\
		\bitbox{8}{Packet Type} & \bitbox{8}{Sequence Number} \\
		\bitbox{16}{Number of Samples} \\
		\bitbox{16}{Sample Period in \textmu s} \\
		\begin{rightwordgroup}{Payload}
			\bitbox{4}{Pad} & \bitbox{12}{Sample 0} \\
			\bitbox{4}{Pad} & \bitbox{12}{Sample 1} \\
			\wordbox[tlr]{1}{\dots} \\
			\wordbox[lr]{1}{Samples} \\
			\wordbox[lr]{1}{\dots} \\
			\bitbox{4}{Pad} & \bitbox{12}{Sample 0} \\
		\end{rightwordgroup}
	\end{bytefield}
	\caption{12 bit mode data packet structure}
\end{figure}

\end{document}
