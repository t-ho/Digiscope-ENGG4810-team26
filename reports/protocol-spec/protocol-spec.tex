\documentclass[]{article}

\usepackage{bytefield}
\usepackage{float}

%opening
\title{ENGG4810 Oscilloscope Communication Protocol Specification}
\author{
	Ryan Fitzimon\\
	43214907\\
	\texttt{ryan.fitzsimon@uq.net.au}
	\and
	Minh Toan Ho\\
	43129560\\
	\texttt{minh.ho2@uq.net.au}
}

\begin{document}

\maketitle

\section{Packet Types}
\begin{table}[H]
	\centering
	\begin{tabular}{|c|c|}
		\hline
		\textbf{Identifier} & \textbf{Type} \\ \hline
		0xD1 & 8 bit sample packet, channel 1 \\ \hline
		0xD2 & 8 bit sample packet, channel 2 \\ \hline
		0xE1 & 12 bit sample packet, channel 1 \\ \hline
		0xE2 & 12 bit sample packet, channel 2 \\ \hline
	\end{tabular}
	\caption{Packet type identifiers}
\end{table}

\section{General Structure}
\begin{figure}[H]
	\centering
	\begin{bytefield}[bitwidth=2em]{16}
		\bitheader{0-15} \\
		\bitbox{8}{Packet Type} & \bitbox[ltr]{8}{} \\
		\wordbox[lr]{1}{\dots} \\
		\wordbox[lr]{1}{Data} \\
		\wordbox[blr]{1}{\dots}
	\end{bytefield}
	\caption{Generic packet structure}
\end{figure}

\section{Command Packet}
\begin{figure}[H]
	\centering
	\begin{bytefield}[bitwidth=2em]{16}
		\bitheader{0-15} \\
		\bitbox{8}{Packet Type} & \bitbox{8}{Command Argument}
	\end{bytefield}
	\caption{Command packet structure}
\end{figure}

\section{Data Packet}
\subsection{8 bit mode}
\begin{figure}[H]
	\centering
	\begin{bytefield}[bitwidth=2em]{16}
		\bitheader{0-15} \\
		\bitbox{8}{Packet Type} & \bitbox{8}{Sequence Number} \\
		\bitbox{16}{Number of Samples}  \\
		\begin{rightwordgroup}{Payload}
			\bitbox{8}{Sample 0} & \bitbox{8}{Sample 1} \\
			\bitbox{8}{Sample 2} & \bitbox{8}{Sample 3}\\
			\wordbox[tlr]{1}{\dots} \\
			\wordbox[lr]{1}{Samples} \\
			\wordbox[lr]{1}{\dots} \\
			\bitbox[blr]{8}{} & \bitbox{8}{Sample n}
		\end{rightwordgroup}
	\end{bytefield}
	\caption{8 bit mode data packet structure}
\end{figure}

\subsection{12 bit mode}
\begin{figure}[H]
	\centering
	\begin{bytefield}[bitwidth=2em]{16}
		\bitheader{0-15} \\
		\bitbox{8}{Packet Type} & \bitbox{8}{Sequence Number} \\
		\bitbox{16}{Number of Samples}  \\
		\begin{rightwordgroup}{Payload}
			\bitbox{12}{Sample 0} & \bitbox{4}{Sample 1\dots} \\
			\bitbox{4}{\dots Sample 1} & \bitbox{12}{Sample 2}\\
			\wordbox[tlr]{1}{\dots} \\
			\wordbox[lr]{1}{Samples} \\
			\wordbox[lr]{1}{\dots} \\
			\bitbox[blr]{4}{} & \bitbox{12}{Sample n}
		\end{rightwordgroup}
	\end{bytefield}
	\caption{12 bit mode data packet structure}
\end{figure}

\end{document}
